\documentclass[11pt]{article}
\usepackage[margin=1in]{geometry}
\usepackage[utf8]{inputenc}
\usepackage[T1]{fontenc}
\usepackage{helvet}
\usepackage{booktabs, tabularx, multirow}
\usepackage{enumitem}
\usepackage{hyperref}
\usepackage{fancyhdr}
\usepackage{graphicx}
\usepackage{xcolor}
\usepackage{minted}
\usepackage{titling}
\usepackage[bottom]{footmisc}
\usepackage{titlesec}
\usepackage{parskip}
\usepackage{amsmath}
\usepackage{listings}

\definecolor{primaryblue}{RGB}{0, 102, 204}
\definecolor{secondarygray}{RGB}{80, 80, 80}
\definecolor{codebg}{RGB}{245, 245, 245}

\hypersetup{
    colorlinks=true,
    linkcolor=primaryblue,
    filecolor=magenta,
    urlcolor=cyan,
    pdftitle={1-Month Expert Architecture Training Plan: Week 1, Day 4},
    bookmarks=true,
}

\pagestyle{fancy}
\fancyhf{}
\rhead{\textbf{1-Month Expert Architecture Training}}
\lhead{\textbf{ISYGO Consulting Services}}
\rfoot{Page \thepage}

\titleformat{\section}{\Large\bfseries\sffamily\color{primaryblue}}{\thesection}{1em}{}
\titleformat{\subsection}{\large\bfseries\sffamily\color{secondarygray}}{\thesubsection}{1em}{}
\titleformat{\subsubsection}{\normalsize\bfseries\sffamily\color{secondarygray}}{\thesubsubsection}{1em}{}

\setlist[itemize]{noitemsep,topsep=2pt}
\setlist[enumerate]{noitemsep,topsep=2pt}

\lstset{
  backgroundcolor=\color{codebg},
  basicstyle=\small\ttfamily,
  frame=single,
  breaklines=true,
}

\begin{document}

\begin{titlepage}
    \centering
    \vspace*{1cm}
    \vspace{0.5cm}
    {\color{primaryblue}\hrule height 2pt}
    \vspace{0.5cm}
    {\LARGE\sffamily\bfseries 1-Month Expert Architecture Training Plan\par}
    \vspace{0.3cm}
    {\large\sffamily\itshape Week 1, Day 4: MLOps \& Model Lifecycle\par}
    \vspace{1.5cm}
    {\large\sffamily Prepared by: Sami Mbarki \\ Solution Architect, Java Expert \\ ISYGO Consulting Services}
    \vspace{1cm}
    {\large\sffamily 18 September 2025}
    \vfill
    \begin{minipage}{0.8\textwidth}
        \centering
        \small\sffamily \textbf{ISYGO Consulting Services} \\
        \small\sffamily Delivering Advanced Architectural Training for Enterprise Solutions \\
        \vspace{0.5cm}
        \small\sffamily Document ID: 28f69273-7825-4ce5-af65-034dd404ca6c \\
        \small\sffamily Version: 1.0 \\
        \small\sffamily Confidential: For Internal Training Use Only
    \end{minipage}
    {\color{primaryblue}\hrule height 2pt}
\end{titlepage}

\pretitle{\begin{center}\LARGE\sffamily\bfseries}
\posttitle{\end{center}}
\preauthor{\begin{center}\large\sffamily}
\postauthor{\end{center}}
\predate{\begin{center}\large\sffamily}
\postdate{\end{center}}

\pagestyle{fancy}
\title{1-Month Expert Training in Microservices and Polyglot Persistence}
\author{Sami Mbarki \\ Solution Architect, Java Expert \\ ISYGO Consulting Services}
\date{18 September 2025}
\maketitle

\tableofcontents
\newpage

\section{Preface: Operationalizing Machine Learning}
This comprehensive chapter on MLOps turns Day 4 into an educational tome, defining terms like "model serving", clarifying lifecycle stages, and exploring technologies (MLflow, Triton) with alternatives (Sagemaker, Kubeflow). Exercises promote hands-on discovery.

\subsection{How to Use This Book}
- Diagram lifecycle for visual learning.
- Compare tools for decision-making.

\section{Introduction: From Models to Production}
What is "technical debt" in ML, and how does MLOps mitigate it? MLOps combines ML, DevOps, and data engineering. Historical: Coined in 2015 by Google in "Hidden Technical Debt in Machine Learning Systems". Alternatives: DataOps for data focus.

\subsection{Learning Objectives}
Including understanding drift detection.

\section{The Machine Learning Lifecycle}
\subsection{Stages in Detail}
1. Business Understanding: Define metrics (e.g., F1-score for classification).
2. Data Engineering: Feature engineering (e.g., TF-IDF for documents).
3. Model Development: Hyperparameter tuning with grid search.
4. Evaluation: Bias detection using fairness metrics.
5. Deployment: A/B testing.
6. Monitoring: Detect data drift (change in input distribution) vs concept drift (change in target relationship).

Diagram:
\begin{figure}[h]
    \centering
    \includegraphics[width=0.8\textwidth]{ml_lifecycle.pdf}
    \caption{ML Lifecycle with Feedback Loops}
\end{figure}

Clarification: CI/CD/CT - Continuous Training for retraining.

Pitfalls: Ignoring drift causes model decay.

\section{MLOps Principles and Tools}
Definition: MLOps is practices for reliable ML deployment.

Terminology: "Feature store" (e.g., Feast) for reusable features.

\subsection{MLflow: End-to-End Management}
Definition: Open-source platform for tracking, packaging, models.

Components: Tracking (log params/metrics), Projects (packaging), Models (serving), Registry (versioning).

Code Example for Tracking:
\begin{minted}[frame=single,fontsize=\small,bgcolor=codebg]{python}
import mlflow
import mlflow.sklearn
from sklearn.model_selection import train_test_split
from sklearn.linear_model import LogisticRegression

mlflow.start_run()
X_train, X_test, y_train, y_test = train_test_split(X, y)
model = LogisticRegression()
model.fit(X_train, y_train)
mlflow.log_param("C", model.C)
mlflow.log_metric("accuracy", model.score(X_test, y_test))
mlflow.sklearn.log_model(model, "model")
mlflow.end_run()
\end{minted}

Alternatives: DVC for version control, Kubeflow for Kubernetes-native.

Best Practices: Use UI for experiment comparison.

Pitfalls: Poor logging leads to irreproducible models.

\section{Model Deployment with Triton}
Definition: Triton is NVIDIA's inference server for multi-framework models.

Terminology: "Dynamic batching" - groups requests for GPU efficiency.

Clarification: Supports ONNX, TensorFlow, PyTorch.

Alternatives: TensorFlow Serving (TF-specific), TorchServe (PyTorch-focused).

Configuration Example:
\lstset{language=yaml}
\begin{lstlisting}
name: "document_model"
backend: "tensorrt"
max_batch_size: 16
input: [
  {
    name: "input"
    data_type: TYPE_FP32
    dims: [ -1, 512 ]
  }
]
output: [
  {
    name: "output"
    data_type: TYPE_FP32
    dims: [ -1, 10 ]
  }
]
dynamic_batching: { }
\end{lstlisting}

Advanced: Model ensemble for chaining (e.g., OCR + extraction).

Case Study: Fraud detection with low-latency inference.

\section{Testing ML Pipelines}
Definition: "Behavioral testing" verifies model fairness.

Terminology: "Shadow deployment" - test new models in parallel.

Alternatives: Great Expectations for data validation.

\section{Lab: MLOps in Practice}
Detailed: Train a simple model, log with MLflow, deploy to Triton.

\section{Wrap-Up: Socratic Discussion}
Discuss: How does concept drift affect our capstone? What alternative to MLflow?

\section{Student Exercises and Review Questions}
1. Log an experiment in MLflow.
2. Compare Triton and TF Serving.

\section{Glossary}
Expanded:
\begin{itemize}
    \item \textbf{Data Drift}: Input change over time.
    \item \textbf{Concept Drift}: Target relationship change.
    \item \textbf{ONNX}: Open Neural Network Exchange format.
    \item \textbf{Dynamic Batching}: Request grouping.
    \item \textbf{Model Registry}: Versioned model store.
\end{itemize}

\section{References and Further Reading}
Books:
\begin{itemize}
    \item Sculley, D., et al. "Hidden Technical Debt in Machine Learning Systems." NIPS 2015.
    \item Baylor, D., et al. "TFX: Production ML Platform." KDD 2017.
    \item Ameisen, E. \textit{Building ML Powered Applications}. O'Reilly, 2020.
\end{itemize}

Online:
\begin{itemize}
    \item MLflow: \url{https://mlflow.org}.
    \item Triton: \url{https://developer.nvidia.com/triton-inference-server}.
\end{itemize}

\end{document}