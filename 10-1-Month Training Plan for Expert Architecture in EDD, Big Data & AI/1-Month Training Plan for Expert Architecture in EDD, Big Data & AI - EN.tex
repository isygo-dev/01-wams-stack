\documentclass[11pt]{article}
\usepackage[margin=1in]{geometry}
\usepackage{tabularx}
\usepackage{booktabs}
\usepackage{multirow}
\usepackage[utf8]{inputenc}
\usepackage[T1]{fontenc}
\usepackage{hyperref}
\usepackage{fancyhdr}
\usepackage{graphicx}
\usepackage{xcolor}
\usepackage[bottom]{footmisc}
\usepackage{times}

\hypersetup{
    colorlinks=true,
    linkcolor=blue,
    filecolor=magenta,
    urlcolor=cyan,
    pdftitle={1-Month Expert Architecture Training},
    bookmarks=true,
}

\pagestyle{fancy}
\fancyhf{}
\rhead{1-Month Expert Architecture Training}
\lhead{ISYGO Consulting Services}
\cfoot{\small Inspire Success, Your Goals \& Opportunities}
\rfoot{Page \thepage}

\begin{document}

% Cover Page
\begin{titlepage}
    \centering
    \vspace*{1cm}
    % Logo placeholder (uncomment and replace with actual logo file)
    % \includegraphics[width=0.3\textwidth]{isygologo.png}
    \vspace{0.5cm}
    {\color{blue}\hrule height 2pt}
    \vspace{0.5cm}
    {\LARGE \textbf{1-Month Expert Architecture Training}\par}
    \vspace{0.3cm}
    {\large \textit{Event-Driven Architecture, Polyglot Persistence, Big Data, AI, and Multitenacy}\par}
    \vspace{1.5cm}
    \begin{tabular}{c}
        \large \textbf{Prepared by: Sami Mbarki} \\
        \large Solution Architect, Java Expert \\
        \large ISYGO Consulting Services \\
    \end{tabular}
    \vspace{1cm}
    {\large 17 September 2025 \par}
    \vfill
    \begin{minipage}{0.8\textwidth}
        \centering
        \small \textbf{ISYGO Consulting Services} \\
        \small \textit{Inspire Success, Your Goals \& Opportunities} \\
        \small Delivering Advanced Architectural Training for Enterprise Solutions \\
        \vspace{0.5cm}
        \small Document ID: ede2b784-641c-41b2-8f8d-d9307e59fe85 \\
        \small Version: 1.0 \\
        \small Confidential: For Internal Training Use Only
    \end{minipage}
    {\color{blue}\hrule height 2pt}
\end{titlepage}

\pagestyle{fancy}
\title{1-Month Expert Training in Microservices and Polyglot Persistence}
\author{Sami Mbarki \\ Solution Architect, Java Expert \\ ISYGO Consulting Services}
\date{17 September 2025}
\maketitle

% Executive Summary
\section*{Executive Summary}
This intensive 1-month program equips experienced software professionals to design scalable, multitenant, event-driven microservice systems with polyglot persistence for document processing. Participants will build a capstone project—an Intelligent Document Processing Platform—that enables single or bulk document uploads, archiving in Apache Cassandra, AI-driven analysis, JSON transformation, and storage in PostgreSQL with tenant isolation. Using technologies like Apache Kafka, Spark, LLaMA 3, Cassandra, and PostgreSQL, the curriculum emphasizes production-ready architectures with robust testing, monitoring, and optimization. With 35 hours of weekly training (7 hours/day), structured resources, and mentorship, participants emerge as expert architects ready for enterprise challenges.

\tableofcontents
\newpage

\section{Program Philosophy \& Objectives}
% Defining the program's core philosophy
This 1-month program transforms software professionals into expert architects capable of designing scalable, multitenant, event-driven microservice systems integrating Event-Driven Design (EDD), Big Data, AI, and polyglot persistence. The curriculum rests on three pillars:
\begin{enumerate}
    \item \textbf{Technological Mastery:} Deep understanding of tools like Apache Kafka, Spark, LLaMA 3, Cassandra, PostgreSQL, and multitenant architectures.
    \item \textbf{Architectural Integration:} Combining EDD, Big Data, AI, microservices, and polyglot persistence into cohesive, resilient systems.
    \item \textbf{Production Readiness:} Prioritizing performance, security, tenant isolation, testing, and monitoring.
\end{enumerate}

% Outlining the capstone project
\textbf{Capstone Project:} Participants will design an \textbf{Intelligent Document Processing Platform}, a microservice-based system enabling tenants to upload documents (single or bulk), archive raw metadata in Cassandra, analyze content using LLaMA 3 for semantic extraction (e.g., invoice numbers), transform extracted data into JSON objects, and store them in PostgreSQL with multitenant isolation. The system uses Kafka for event-driven orchestration, with Cassandra for high-volume archival and PostgreSQL for structured storage, supported by comprehensive testing, monitoring, and optimization.

% Adding visual context
\begin{figure}[h]
    \centering
    \includegraphics[width=0.8\textwidth]{capstone_architecture.pdf}
    \caption{Event-Driven Microservice Architecture with Polyglot Persistence}
\end{figure}

% Transition to next section
This philosophy guides the selection of target audience and prerequisites, ensuring participants are equipped to build the capstone system.

\section{Target Audience \& Prerequisites}

% Specifying the intended participants
\subsection{Target Audience}
\begin{itemize}
    \item Senior Software Engineers transitioning to architecture roles
    \item Data Engineers taking on architectural responsibilities
    \item DevOps Engineers building scalable data/AI infrastructure
    \item Solutions Architects deepening expertise in multitenant systems
\end{itemize}

% Listing required skills
\subsection{Technical Prerequisites}
\begin{itemize}
    \item Proficiency in Java, Scala, or Python
    \item Strong understanding of distributed systems (e.g., CAP theorem)
    \item Advanced SQL and NoSQL concepts (e.g., indexing, keyspaces)
    \item Familiarity with Linux/Unix, Git, and CI/CD pipelines
    \item Working knowledge of cloud platforms (AWS, Azure, or GCP)
    \item 3-5 years of software development experience
\end{itemize}

% Adding resources for preparation
\subsection{Recommended Resources}
\begin{itemize}
    \item \textit{Book:} \textit{Designing Data-Intensive Applications} by Martin Kleppmann
    \item \textit{Book:} \textit{Building Microservices} by Sam Newman
    \item \textit{Online:} Confluent Kafka Fundamentals (\url{https://confluent.io/learn})
    \item \textit{Online:} DataStax Academy Cassandra Fundamentals (\url{https://www.datastax.com/learn})
    \item \textit{Online:} Microsoft Azure Microservices Guide (\url{https://learn.microsoft.com/en-us/azure/architecture/microservices})
    \item \textit{Online:} Data Modeling in Apache Cassandra (DataStax, \url{https://www.datastax.com/learn/cassandra-data-modeling})
    \item \textit{Online:} PostgreSQL JSONB Guide (\url{https://www.postgresql.org/docs/current/datatype-json.html})
\end{itemize}

% Transition
These prerequisites prepare participants for the core technologies and hands-on labs detailed next, building toward the polyglot, microservice-based capstone project.

\section{Core Technologies \& Tools}

% Detailing EDD technologies
\subsection{Event-Driven Design Stack}
\begin{itemize}
    \item \textbf{Apache Kafka:} Core platform, Connect, Streams, Schema Registry
    \item \textbf{Debezium:} Real-time change data capture
    \item \textbf{Avro/Protobuf:} Schema-driven serialization
\end{itemize}

% Detailing Big Data technologies
\subsection{Big Data Processing Framework}
\begin{itemize}
    \item \textbf{Apache Spark:} Large-scale data processing with Structured Streaming
    \item \textbf{Delta Lake/Iceberg:} ACID-compliant table formats
    \item \textbf{Medallion Architecture:} Bronze, Silver, Gold layers
\end{itemize}

% Detailing AI/ML technologies
\subsection{AI/ML Engineering Ecosystem}
\begin{itemize}
    \item \textbf{MLflow:} ML lifecycle management
    \item \textbf{Scikit-learn/TensorFlow/PyTorch:} Model development
    \item \textbf{Feast:} Feature store for multitenant features
    \item \textbf{Triton Inference Server:} Model serving
\end{itemize}

% Detailing orchestration and infrastructure
\subsection{Orchestration \& Infrastructure}
\begin{itemize}
    \item \textbf{Kubernetes:} Container orchestration
    \item \textbf{Docker:} Containerization
    \item \textbf{FastAPI:} RESTful APIs for document upload
    \item \textbf{Cloud Platforms:} AWS (S3, EMR, EKS), Azure (ADLS, Databricks, AKS), GCP (GCS, Dataproc, GKE)
\end{itemize}

% Detailing capstone-specific components
\subsection{Capstone Practice Components}
\begin{itemize}
    \item \textbf{Apache Camel:} Document routing
    \item \textbf{Drools:} Business rules for JSON validation
    \item \textbf{Ollama/LLaMA 3:} Semantic analysis
    \item \textbf{PostgreSQL:} Structured storage with JSONB
    \item \textbf{Apache Cassandra:} High-volume archival storage
    \item \textbf{Spring State Machine:} Document lifecycle management
    \item \textbf{Tesseract OCR:} Text extraction
    \item \textbf{Redis:} Distributed caching
    \item \textbf{Keycloak:} Multitenant authentication
    \item \textbf{Prometheus/Grafana:} System monitoring
    \item \textbf{JUnit/JMeter/K6:} Testing frameworks
    \item \textbf{OpenLineage:} Data lineage tracking
\end{itemize}

% Transition
These technologies form the foundation for the training plan, enabling participants to build a production-ready, microservice-based, polyglot document processing system.

\section{Detailed Training Plan}
% Structuring the training schedule
\textbf{Structure:} Each week includes 5 days of 7 hours (3h theory, 3.5h labs, 0.5h wrap-up), totaling 35 hours weekly. Labs build a multitenant, event-driven microservice system with polyglot persistence, supported by testing, monitoring, and optimization. Microservice architecture and polyglot persistence concepts are integrated into theory and labs to support the capstone project.

\begin{figure}[h]
    \centering
    \includegraphics[width=\textwidth]{gantt_chart.pdf}
    \caption{4-Week Training Schedule}
\end{figure}

\subsection{Week 1: Foundational Mastery \& Ingestion Layer}
% Setting the week's goal
\textbf{Goal:} Establish event-driven ingestion for document uploads (single/bulk) with PostgreSQL for initial storage, testing, and monitoring.

\subsubsection{Day 1: Kafka \& Event-Driven Foundations}
\textbf{Objective:} Master Kafka fundamentals, security, and event-driven patterns for multitenant document uploads with initial testing and monitoring.

\textbf{Theory (3h):}
\begin{itemize}
    \item Kafka architecture: Brokers, Topics, Partitions, ISR (1h)
    \item Kafka security and Schema Registry: SSL/TLS, ACLs, Avro with tenant metadata (1h)
    \item Testing and monitoring: Unit tests (JUnit), Prometheus metrics for Kafka (1h)
\end{itemize}
\textbf{Resources:}
\begin{itemize}
    \item \textit{Book:} \textit{Kafka: The Definitive Guide} (Ch. 1-3, 7)
    \item \textit{Online:} Confluent Kafka Security Tutorial (\url{https://docs.confluent.io/platform/current/security})
    \item \textit{Video:} Introduction to Microservice Testing (YouTube, 30-min pre-watch)
\end{itemize}

\textbf{Lab (3.5h):} Deploy a secure Kafka cluster for document uploads
\begin{itemize}
    \item \textit{Objective:} Set up a production-grade Kafka cluster with security, unit tests, and monitoring for multitenant uploads.
    \item \textit{Steps:}
        \begin{enumerate}
            \item Deploy Kafka using Docker on Minikube.
            \item Configure SSL/TLS and ACLs for tenant-specific topics (e.g., tenantA-documents).
            \item Set up Schema Registry with Avro schemas embedding tenant IDs.
            \item Develop a producer to publish single document upload events.
            \item Write JUnit tests for producer/consumer logic.
            \item Configure Prometheus for Kafka metrics (e.g., topic lag).
            \item Complete Knowledge Check: Quiz on ACLs, schema validation, and unit testing.
        \end{enumerate}
    \item \textit{Technologies:} Kafka, Docker, Minikube, Avro, JUnit, Prometheus
    \item \textit{Support:} Sample PDFs, JUnit templates, Prometheus config.
\end{itemize}

\textbf{Wrap-Up (0.5h):}
\begin{itemize}
    \item Review test results and Kafka metrics.
    \item Q\&A on multitenant Kafka design.
    \item \textit{Capstone Progress:} Initialized document upload pipeline.
\end{itemize}

\subsubsection{Day 2: Microservices \& Bulk Uploads}
\textbf{Objective:} Understand microservice architecture and implement bulk document uploads with integration tests and monitoring.

\textbf{Theory (3h):}
\begin{itemize}
    \item Microservice architecture: Definition, pros (scalability, fault isolation), cons (complexity, latency), utility in document processing (0.5h)
    \item Kafka Connect for bulk uploads (1h)
    \item Spark architecture: Driver/Executors, Catalyst Optimizer (1h)
    \item Integration testing for microservice pipelines (0.5h)
\end{itemize}
\textbf{Resources:}
\begin{itemize}
    \item \textit{Book:} \textit{Building Microservices} by Sam Newman (Ch. 1-2)
    \item \textit{Online:} Microsoft Azure Microservices Guide (\url{https://learn.microsoft.com/en-us/azure/architecture/microservices})
\end{itemize}

\textbf{Lab (3.5h):} Build microservice-based bulk upload pipeline
\begin{itemize}
    \item \textit{Objective:} Implement bulk document uploads using a FastAPI microservice with integration tests and monitoring.
    \item \textit{Steps:}
        \begin{enumerate}
            \item Deploy Kafka Connect with a file source connector for bulk PDFs.
            \item Configure tenant-specific topics for bulk uploads.
            \item Develop a FastAPI microservice to handle bulk uploads to Kafka.
            \item Set up a Spark cluster (Docker or EMR/Dataproc).
            \item Write a Spark job to read bulk upload events.
            \item Write integration tests for Kafka-to-Spark microservice flow.
            \item Monitor Spark job and microservice metrics with Prometheus.
            \item Complete Knowledge Check: Quiz on microservice benefits and trade-offs.
        \end{enumerate}
    \item \textit{Technologies:} Kafka Connect, FastAPI, Spark, JUnit, Prometheus
    \item \textit{Support:} Sample bulk PDF dataset, FastAPI template, test templates.
\end{itemize}
\textbf{Wrap-Up (0.5h):}
\begin{itemize}
    \item Review integration tests and metrics.
    \item Q\&A on microservice-based uploads.
    \item \textit{Capstone Progress:} Enabled bulk document uploads via microservice.
\end{itemize}

\subsubsection{Day 3: Spark Streaming \& Delta Lake}
\textbf{Theory (3h):}
\begin{itemize}
    \item Structured Streaming, watermarking (1h)
    \item Medallion Architecture with Delta Lake (1h)
    \item Spark optimization: Partitioning, caching (1h)
\end{itemize}
\textbf{Lab (3.5h):} Archive documents in Delta Lake
\begin{itemize}
    \item \textit{Objective:} Stream document uploads to Delta Lake with optimization and monitoring.
    \item \textit{Steps:}
        \begin{enumerate}
            \item Develop a Spark Streaming job for document events.
            \item Implement watermarking for late uploads.
            \item Archive to Delta Lake Bronze layer with tenant partitions.
            \item Optimize Spark job with caching.
            \item Test data integrity with JUnit tests.
            \item Monitor job performance with Prometheus.
            \item Complete Knowledge Check on partitioning and monitoring.
        \end{enumerate}
    \item \textit{Technologies:} Spark Streaming, Delta Lake, JUnit, Prometheus
    \item \textit{Support:} Sample dataset, optimization guide.
\end{itemize}
\textbf{Wrap-Up (0.5h):}
\begin{itemize}
    \item Discuss optimization and monitoring results.
    \item \textit{Capstone Progress:} Archived documents in Delta Lake.
\end{itemize}

\subsubsection{Day 4: MLOps \& Model Lifecycle}
\textbf{Theory (3h):}
\begin{itemize}
    \item ML lifecycle: CI/CD, MLflow (1h)
    \item Model deployment: Triton Server (1h)
    \item Testing ML pipelines (1h)
\end{itemize}
\textbf{Lab (3.5h):} Mock AI analysis for documents
\begin{itemize}
    \item \textit{Objective:} Deploy a model for document classification with testing.
    \item \textit{Steps:}
        \begin{enumerate}
            \item Install MLflow and configure tracking.
            \item Train a Scikit-learn model for document classification.
            \item Log multitenant artifacts in MLflow.
            \item Deploy model via Triton Server.
            \item Write unit tests for model inference.
            \item Monitor inference latency with Prometheus.
            \item Complete Knowledge Check on ML testing.
        \end{enumerate}
    \item \textit{Technologies:} MLflow, Scikit-learn, Triton, JUnit, Prometheus
    \item \textit{Support:} Pre-built model template.
\end{itemize}
\textbf{Wrap-Up (0.5h):}
\begin{itemize}
    \item Review ML tests and metrics.
    \item \textit{Capstone Progress:} Simulated AI analysis.
\end{itemize}

\subsubsection{Day 5: Capstone Ingestion Pipeline}
\textbf{Theory (3h):}
\begin{itemize}
    \item Multitenant ingestion patterns with microservices (1h)
    \item Monitoring strategies: Grafana dashboards (1h)
    \item PostgreSQL for JSON storage (1h)
\end{itemize}
\textbf{Lab (3.5h):} Deploy microservice-based ingestion and PostgreSQL storage
\begin{itemize}
    \item \textit{Objective:} Integrate upload and storage with microservices, testing, and monitoring.
    \item \textit{Steps:}
        \begin{enumerate}
            \item Develop a FastAPI microservice for document uploads to Kafka.
            \item Integrate Kafka and Spark for processing upload events.
            \item Store upload metadata in PostgreSQL (JSONB) with tenant schemas.
            \item Write integration tests for the microservice pipeline.
            \item Set up Grafana dashboards for pipeline metrics.
            \item Optimize Kafka partitions for throughput.
            \item Complete Knowledge Check on microservice ingestion.
        \end{enumerate}
    \item \textit{Technologies:} FastAPI, Kafka, Spark, PostgreSQL, Grafana
    \item \textit{Support:} Sample PDFs, PostgreSQL schema template, FastAPI template.
\end{itemize}
\textbf{Wrap-Up (0.5h):}
\begin{itemize}
    \item Review test results and dashboards.
    \item Q\&A on microservice-based ingestion.
    \item \textit{Capstone Progress:} Deployed microservice for document uploads and PostgreSQL storage.
\end{itemize}

\textbf{Week in Review:}
\begin{itemize}
    \item \textit{Deliverables:} Multitenant upload (single/bulk) pipeline with microservices, PostgreSQL storage, unit/integration tests, Prometheus/Grafana monitoring, optimized Kafka partitions.
    \item \textit{Milestone:} Submit capstone progress reflection.
\end{itemize}

\subsection{Week 2: Architectural Integration}
\textbf{Goal:} Implement AI analysis, JSON transformation, and Cassandra archival with microservices, polyglot persistence, testing, and monitoring.

\subsubsection{Day 1: Polyglot Persistence \& Feature Stores}
\textbf{Objective:} Understand polyglot persistence and implement a feature store for document metadata with testing and monitoring.

\textbf{Theory (3h):}
\begin{itemize}
    \item Polyglot persistence: Definition, pros (performance, scalability), cons (consistency, complexity), utility in document processing (0.5h)
    \item Feature store architecture: Multitenant serving (1h)
    \item Real-time feature computation with microservices (1h)
    \item Testing feature pipelines (0.5h)
\end{itemize}
\textbf{Resources:}
\begin{itemize}
    \item \textit{Online:} Data Modeling in Apache Cassandra (\url{https://www.datastax.com/learn/cassandra-data-modeling})
    \item \textit{Online:} PostgreSQL JSONB Guide (\url{https://www.postgresql.org/docs/current/datatype-json.html})
\end{itemize}

\textbf{Lab (3.5h):} Store document metadata in Feast with polyglot persistence
\begin{itemize}
    \item \textit{Objective:} Implement a feature store with PostgreSQL and introduce Cassandra concepts, including testing and monitoring.
    \item \textit{Steps:}
        \begin{enumerate}
            \item Deploy Feast with PostgreSQL backend for tenant-specific features.
            \item Define multitenant feature views for document metadata.
            \item Build Spark streaming job for feature computation.
            \item Set up Cassandra with tenant-specific keyspaces for future archival.
            \item Write integration tests for feature pipeline and polyglot setup.
            \item Monitor feature store and Cassandra with Prometheus.
            \item Complete Knowledge Check: Quiz on polyglot persistence (Cassandra vs. PostgreSQL use cases).
        \end{enumerate}
    \item \textit{Technologies:} Feast, Spark, PostgreSQL, Cassandra, JUnit, Prometheus
    \item \textit{Support:} Feature view templates, Cassandra keyspace template.
\end{itemize}
\textbf{Wrap-Up (0.5h):}
\begin{itemize}
    \item Review tests and metrics.
    \item Q\&A on polyglot persistence.
    \item \textit{Capstone Progress:} Stored document metadata in Feast and initialized Cassandra.
\end{itemize}

\subsubsection{Day 2: Inference Architecture}
\textbf{Objective:} Implement an AI analysis microservice with best practices, testing, and monitoring.

\textbf{Theory (3h):}
\begin{itemize}
    \item Microservice best practices: DDD, API gateways, event-driven communication (1h)
    \item Low-latency serving with caching (1h)
    \item Monitoring microservices with Grafana (1h)
\end{itemize}
\textbf{Lab (3.5h):} Build AI analysis microservice
\begin{itemize}
    \item \textit{Objective:} Develop a FastAPI microservice for LLaMA 3 analysis using microservice best practices, with testing and monitoring.
    \item \textit{Steps:}
        \begin{enumerate}
            \item Create FastAPI microservice for tenant-specific analysis with DDD.
            \item Deploy on Kubernetes with Istio for API Gateway.
            \item Integrate LLaMA 3 for semantic extraction (e.g., invoice numbers).
            \item Write unit tests (JUnit) for FastAPI endpoints.
            \item Monitor API latency with Grafana.
            \item Complete Knowledge Check: Quiz on microservice design patterns (e.g., API Gateway benefits).
        \end{enumerate}
    \item \textit{Technologies:} FastAPI, Kubernetes, Istio, LLaMA 3, JUnit, Grafana
    \item \textit{Support:} Pre-built FastAPI template, Istio configuration.
\end{itemize}
\textbf{Wrap-Up (0.5h):}
\begin{itemize}
    \item Review test results and dashboards.
    \item Q\&A on microservice best practices.
    \item \textit{Capstone Progress:} Enabled AI-driven document analysis microservice.
\end{itemize}

\subsubsection{Day 3: Orchestration Patterns}
\textbf{Theory (3h):}
\begin{itemize}
    \item Workflow orchestration with Spring State Machine (1h)
    \item Apache Camel for microservice integration (1h)
    \item Cassandra for high-volume archival with polyglot best practices (1h)
\end{itemize}
\textbf{Lab (3.5h):} Orchestrate analysis and Cassandra archival
\begin{itemize}
    \item \textit{Objective:} Orchestrate document analysis and archive metadata in Cassandra using microservices and polyglot persistence best practices.
    \item \textit{Steps:}
        \begin{enumerate}
            \item Define states (UPLOADED, ANALYZED, ARCHIVED) in Spring State Machine.
            \item Implement multitenant state transitions.
            \item Optimize Cassandra keyspaces for write-heavy workloads (e.g., tenant-specific partition keys).
            \item Create Camel routes in a microservice to archive metadata in Cassandra.
            \item Write integration tests for archival flow.
            \item Monitor Cassandra with Prometheus.
            \item Complete Knowledge Check on Cassandra archival and polyglot best practices.
        \end{enumerate}
    \item \textit{Technologies:} Spring State Machine, Apache Camel, Cassandra, JUnit, Prometheus
    \item \textit{Support:} Pre-configured Cassandra cluster, sample keyspaces, Camel routes.
\end{itemize}
\textbf{Wrap-Up (0.5h):}
\begin{itemize}
    \item Discuss archival and test results.
    \item Q\&A on polyglot persistence with Cassandra.
    \item \textit{Capstone Progress:} Archived metadata in Cassandra via microservice.
\end{itemize}

\subsubsection{Day 4: Rules Engine \& DDD}
\textbf{Theory (3h):}
\begin{itemize}
    \item Drools for JSON validation (1h)
    \item Domain-Driven Design: Bounded contexts for microservices (1h)
    \item Optimizing Drools performance (1h)
\end{itemize}
\textbf{Lab (3.5h):} Validate JSON output
\begin{itemize}
    \item \textit{Objective:} Apply multitenant JSON validation with optimization in a microservice.
    \item \textit{Steps:}
        \begin{enumerate}
            \item Define Drools rules for JSON validation in a microservice.
            \item Integrate with Camel for routing.
            \item Optimize Drools rule execution.
            \item Write unit tests for validation logic.
            \item Monitor rule performance with Prometheus.
            \item Complete Knowledge Check on DDD and optimization.
        \end{enumerate}
    \item \textit{Technologies:} Drools, Apache Camel, JUnit, Prometheus
    \item \textit{Support:} Sample Drools rules, DDD templates.
\end{itemize}
\textbf{Wrap-Up (0.5h):}
\begin{itemize}
    \item Review validation and optimization.
    \item \textit{Capstone Progress:} Validated JSON objects in microservice.
\end{itemize}

\subsubsection{Day 5: Capstone Processing Core}
\textbf{Theory (3h):}
\begin{itemize}
    \item Multitenant AI and JSON integration in microservices (1h)
    \item Polyglot persistence: Cassandra and PostgreSQL integration, synchronization best practices (1h)
    \item Monitoring with Grafana dashboards (1h)
\end{itemize}
\textbf{Lab (3.5h):} Integrate AI, JSON, and polyglot storage
\begin{itemize}
    \item \textit{Objective:} Store JSON in PostgreSQL and archive in Cassandra with microservices, testing, and monitoring, applying polyglot best practices.
    \item \textit{Steps:}
        \begin{enumerate}
            \item Integrate FastAPI microservice with LLaMA 3 for analysis.
            \item Transform data to JSON via Camel microservice.
            \item Store JSON in PostgreSQL with tenant schemas, using JSONB indexes.
            \item Archive metadata in Cassandra tenant-specific keyspaces.
            \item Use Kafka Connect for data synchronization between Cassandra and PostgreSQL.
            \item Write integration tests for polyglot storage.
            \item Set up Grafana dashboards for pipeline and database metrics.
            \item Complete Knowledge Check on polyglot persistence integration (e.g., synchronization strategies).
        \end{enumerate}
    \item \textit{Technologies:} FastAPI, LLaMA 3, Camel, PostgreSQL, Cassandra, Kafka Connect, Grafana
    \item \textit{Support:} PostgreSQL/Cassandra templates, dashboard configs.
\end{itemize}
\textbf{Wrap-Up (0.5h):}
\begin{itemize}
    \item Review test results and dashboards.
    \item Q\&A on polyglot persistence integration.
    \item \textit{Capstone Progress:} Integrated polyglot storage with microservices.
\end{itemize}

\textbf{Week in Review:}
\begin{itemize}
    \item \textit{Deliverables:} Multitenant AI analysis, JSON transformation, Cassandra archival, PostgreSQL storage, integration tests, Grafana dashboards, all using microservices and polyglot persistence.
    \item \textit{Milestone:} Submit capstone progress reflection.
\end{itemize}

\subsection{Week 3: Optimization, Advanced AI, and Multitenancy}
\textbf{Goal:} Optimize the polyglot document processing system with advanced testing and monitoring.

\subsubsection{Day 1: Performance Optimization}
\textbf{Theory (3h):}
\begin{itemize}
    \item Spark optimization: AQE, skew handling (1h)
    \item Kafka and Cassandra tuning for microservices (1h)
    \item End-to-end pipeline testing (1h)
\end{itemize}
\textbf{Lab (3.5h):} Optimize bulk processing
\begin{itemize}
    \item \textit{Objective:} Enhance throughput for bulk uploads with testing.
    \item \textit{Steps:}
        \begin{enumerate}
            \item Tune Spark jobs with AQE for bulk processing.
            \item Optimize Kafka topics and Cassandra write throughput.
            \item Write end-to-end tests for upload pipeline.
            \item Monitor performance with Prometheus/Grafana.
            \item Complete Knowledge Check on optimization.
        \end{enumerate}
    \item \textit{Technologies:} Spark, Kafka, Cassandra, JUnit, Grafana
    \item \textit{Support:} Sample bulk dataset, optimization guide.
\end{itemize}
\textbf{Wrap-Up (0.5h):}
\begin{itemize}
    \item Discuss optimization and test results.
    \item \textit{Capstone Progress:} Optimized bulk upload performance.
\end{itemize}

\subsubsection{Day 2: Advanced AI \& LLM}
\textbf{Theory (3h):}
\begin{itemize}
    \item RAG architecture with vector embeddings (1h)
    \item Vector databases: PGVector (1h)
    \item Monitoring AI microservices (1h)
\end{itemize}
\textbf{Lab (3.5h):} Implement semantic search for documents
\begin{itemize}
    \item \textit{Objective:} Enable tenant-specific semantic search with monitoring.
    \item \textit{Steps:}
        \begin{enumerate}
            \item Set up PGVector for tenant-specific vectors in PostgreSQL.
            \item Generate embeddings with LLaMA 3 in a microservice.
            \item Implement semantic search with PGVector.
            \item Write tests for search accuracy.
            \item Monitor AI latency with Grafana.
            \item Complete Knowledge Check on AI monitoring.
        \end{enumerate}
    \item \textit{Technologies:} PGVector, LLaMA 3, JUnit, Grafana
    \item \textit{Support:} Sample embeddings.
\end{itemize}
\textbf{Wrap-Up (0.5h):}
\begin{itemize}
    \item Review search and monitoring results.
    \item \textit{Capstone Progress:} Added semantic search capabilities.
\end{itemize}

\subsubsection{Day 3: Model Optimization \& Multi-Modal}
\textbf{Theory (3h):}
\begin{itemize}
    \item Model optimization: Quantization, pruning (1h)
    \item Multi-modal AI for text/image in microservices (1h)
    \item Testing multi-modal pipelines (1h)
\end{itemize}
\textbf{Lab (3.5h):} Enhance multi-modal document analysis
\begin{itemize}
    \item \textit{Objective:} Optimize AI for text/image analysis with testing.
    \item \textit{Steps:}
        \begin{enumerate}
            \item Quantize LLaMA 3 model for efficiency.
            \item Integrate Tesseract OCR with LLaMA 3 in a microservice.
            \item Update JSON transformation for multi-modal data.
            \item Write tests for multi-modal output.
            \item Monitor AI performance with Prometheus.
            \item Complete Knowledge Check on model optimization.
        \end{enumerate}
    \item \textit{Technologies:} LLaMA 3, Tesseract, JUnit, Prometheus
    \item \textit{Support:} Sample image-based PDFs.
\end{itemize}
\textbf{Wrap-Up (0.5h):}
\begin{itemize}
    \item Discuss optimization and test results.
    \item \textit{Capstone Progress:} Enhanced AI for multi-modal documents.
\end{itemize}

\subsubsection{Day 4: Security, Governance, and Multitenancy}
\textbf{Theory (3h):}
\begin{itemize}
    \item Data governance: GDPR, audit trails (1h)
    \item Security: mTLS, OAuth2, RBAC for microservices and polyglot databases (1h)
    \item Polyglot persistence: Securing Cassandra/PostgreSQL (1h)
\end{itemize}
\textbf{Lab (3.5h):} Secure polyglot storage
\begin{itemize}
    \item \textit{Objective:} Secure Cassandra and PostgreSQL with tenant isolation in microservices, applying polyglot best practices.
    \item \textit{Steps:}
        \begin{enumerate}
            \item Implement audit trails with OpenLineage for polyglot data flows.
            \item Deploy Keycloak for OAuth2 authentication in microservices.
            \item Configure RBAC for microservices and databases.
            \item Secure Cassandra keyspaces and PostgreSQL schemas with tenant isolation.
            \item Write security tests for tenant isolation.
            \item Monitor lineage with OpenLineage.
            \item Complete Knowledge Check on polyglot security and microservices.
        \end{enumerate}
    \item \textit{Technologies:} OpenLineage, Keycloak, PostgreSQL, Cassandra
    \item \textit{Support:} Pre-configured Keycloak, Cassandra keyspaces, PostgreSQL schemas.
\end{itemize}
\textbf{Wrap-Up (0.5h):}
\begin{itemize}
    \item Review security and lineage.
    \item Q\&A on securing polyglot persistence.
    \item \textit{Capstone Progress:} Secured polyglot storage with microservices.
\end{itemize}

\subsubsection{Day 5: Observability \& Capstone Enhancement}
\textbf{Theory (3h):}
\begin{itemize}
    \item Advanced monitoring with Prometheus/Grafana for microservices and polyglot databases (1h)
    \item Multitenant AI enhancements (1h)
    \item Optimizing Cassandra/PostgreSQL (1h)
\end{itemize}
\textbf{Lab (3.5h):} Enhance observability
\begin{itemize}
    \item \textit{Objective:} Monitor polyglot document processing with optimization in microservices.
    \item \textit{Steps:}
        \begin{enumerate}
            \item Deploy Prometheus/Grafana for tenant-aware microservice and database monitoring.
            \item Implement OpenLineage for JSON data lineage across Cassandra/PostgreSQL.
            \item Enhance LLaMA 3 with tenant-specific prompts in a microservice.
            \item Optimize Cassandra queries and PostgreSQL JSONB indexes.
            \item Test storage performance with JUnit.
            \item Complete Knowledge Check on observability and polyglot persistence.
        \end{enumerate}
    \item \textit{Technologies:} Prometheus, Grafana, OpenLineage, LLaMA 3, Cassandra, PostgreSQL
    \item \textit{Support:} Sample dashboard templates, Cassandra/PostgreSQL optimization guides.
\end{itemize}
\textbf{Wrap-Up (0.5h):}
\begin{itemize}
    \item Validate dashboards and optimization.
    \item \textit{Capstone Progress:} Added observability to polyglot pipeline with microservices.
\end{itemize}

\textbf{Week in Review:}
\begin{itemize}
    \item \textit{Deliverables:} Optimized polyglot processing, secure storage, end-to-end tests, OpenLineage integration, tenant-specific dashboards, all using microservices and polyglot persistence.
    \item \textit{Milestone:} Submit capstone progress reflection.
\end{itemize}

\subsection{Week 4: Production Readiness \& Finalization}
\textbf{Goal:} Finalize the production-ready polyglot document processing system with microservices.

\subsubsection{Day 1: Load Testing \& Optimization}
\textbf{Theory (3h):}
\begin{itemize}
    \item Load/stress testing with JMeter/K6 for microservices (1h)
    \item Cloud cost management for polyglot systems (1h)
    \item Kubernetes auto-scaling for microservices (1h)
\end{itemize}
\textbf{Lab (3.5h):} Test bulk document processing
\begin{itemize}
    \item \textit{Objective:} Ensure scalability for bulk uploads with load testing, applying microservice best practices.
    \item \textit{Steps:}
        \begin{enumerate}
            \item Run load tests with JMeter/K6 for tenant-specific microservices.
            \item Analyze bottlenecks in upload/AI/storage microservices.
            \item Implement Kubernetes auto-scaling for microservices.
            \item Monitor load test metrics with Grafana.
            \item Complete Knowledge Check on load testing microservices and scalability use cases (e.g., tenant-specific processing).
        \end{enumerate}
    \item \textit{Technologies:} JMeter, K6, Kubernetes, Grafana
    \item \textit{Support:} Sample load test scripts.
\end{itemize}
\textbf{Wrap-Up (0.5h):}
\begin{itemize}
    \item Discuss scalability and test results.
    \item \textit{Capstone Progress:} Validated bulk processing scalability with microservices.
\end{itemize}

\subsubsection{Day 2: Security Hardening}
\textbf{Theory (3h):}
\begin{itemize}
    \item Security hardening for polyglot microservices (1h)
    \item GDPR compliance checks for Cassandra/PostgreSQL (1h)
    \item Monitoring security metrics for microservices and databases (1h)
\end{itemize}
\textbf{Lab (3.5h):} Harden polyglot storage
\begin{itemize}
    \item \textit{Objective:} Secure Cassandra and PostgreSQL with monitoring in microservices, applying polyglot best practices.
    \item \textit{Steps:}
        \begin{enumerate}
            \item Encrypt JSON data in PostgreSQL and Cassandra.
            \item Harden Kubernetes with pod security policies for microservices.
            \item Conduct GDPR compliance checks for polyglot databases.
            \item Write security tests for tenant isolation in microservices and databases.
            \item Monitor security metrics with Prometheus.
            \item Complete Knowledge Check on security testing and polyglot persistence.
        \end{enumerate}
    \item \textit{Technologies:} Kubernetes, Keycloak, PostgreSQL, Cassandra, Prometheus
    \item \textit{Support:} Pre-configured security policies, encryption templates.
\end{itemize}
\textbf{Wrap-Up (0.5h):}
\begin{itemize}
    \item Review security measures and metrics.
    \item \textit{Capstone Progress:} Hardened polyglot system security with microservices.
\end{itemize}

\subsubsection{Day 3: Documentation \& Runbooks}
\textbf{Theory (3h):}
\begin{itemize}
    \item Architecture documentation for microservices and polyglot persistence (1h)
    \item Multitenant runbook creation (1h)
    \item Documenting polyglot test/monitoring setups (1h)
\end{itemize}
\textbf{Lab (3.5h):} Document the polyglot system
\begin{itemize}
    \item \textit{Objective:} Produce documentation for the microservice-based system with polyglot persistence.
    \item \textit{Steps:}
        \begin{enumerate}
            \item Create architecture diagrams with draw.io for microservices and Cassandra/PostgreSQL.
            \item Write runbooks for tenant-specific microservice and database operations.
            \item Document test cases and monitoring for Cassandra/PostgreSQL.
            \item Review with peers; validate with Knowledge Check.
        \end{enumerate}
    \item \textit{Technologies:} draw.io, LaTeX, Markdown
    \item \textit{Support:} Diagram templates, runbook samples.
\end{itemize}
\textbf{Wrap-Up (0.5h):}
\begin{itemize}
    \item Discuss documentation quality.
    \item \textit{Capstone Progress:} Documented polyglot microservice system.
\end{itemize}

\subsubsection{Day 4: Performance Baselines}
\textbf{Theory (3h):}
\begin{itemize}
    \item Defining multitenant SLAs for microservices and polyglot databases (1h)
    \item Benchmarking polyglot microservices (1h)
    \item Optimizing Cassandra/PostgreSQL performance (1h)
\end{itemize}
\textbf{Lab (3.5h):} Establish polyglot system baselines
\begin{itemize}
    \item \textit{Objective:} Set performance baselines for tenant workloads in microservices and polyglot databases, applying optimization best practices.
    \item \textit{Steps:}
        \begin{enumerate}
            \item Define SLAs for upload/AI/storage microservices and Cassandra/PostgreSQL.
            \item Run benchmarks with JMeter/K6 for Cassandra queries and PostgreSQL JSONB.
            \item Optimize Cassandra compaction strategies and PostgreSQL GIN indexes.
            \item Monitor benchmarks with Grafana dashboards.
            \item Complete Knowledge Check on SLAs and polyglot optimization use cases (e.g., e-commerce, IoT).
        \end{enumerate}
    \item \textit{Technologies:} JMeter, K6, Cassandra, PostgreSQL, Grafana
    \item \textit{Support:} Sample benchmark scripts, optimization guides.
\end{itemize}
\textbf{Wrap-Up (0.5h):}
\begin{itemize}
    \item Review baseline results.
    \item \textit{Capstone Progress:} Established polyglot performance baselines with microservices.
\end{itemize}

\subsubsection{Day 5: Final Presentation}
\textbf{Theory (3h):}
\begin{itemize}
    \item Presentation skills for architects (1h)
    \item Defending microservice and polyglot design decisions (1h)
    \item Presenting test and monitoring results (1h)
\end{itemize}
\textbf{Lab (3.5h):} Present the polyglot system
\begin{itemize}
    \item \textit{Objective:} Deliver a demo of the microservice-based document processing system with polyglot persistence.
    \item \textit{Steps:}
        \begin{enumerate}
            \item Prepare a live demo of upload, AI, Cassandra archival, and PostgreSQL storage microservices.
            \item Create slides highlighting microservice architecture, polyglot persistence, tests, and monitoring.
            \item Rehearse with Q\&A.
            \item Present to peers/instructors.
            \item Complete Knowledge Check on presentation clarity.
        \end{enumerate}
    \item \textit{Technologies:} PowerPoint, Kubernetes, FastAPI, Cassandra, PostgreSQL
    \item \textit{Support:} Presentation template.
\end{itemize}
\textbf{Wrap-Up (0.5h):}
\begin{itemize}
    \item Finalize deliverables.
    \item \textit{Capstone Progress:} Presented complete polyglot microservice system.
\end{itemize}

\textbf{Week in Review:}
\begin{itemize}
    \item \textit{Deliverables:} Production-ready polyglot document processing system with microservices, load tests, tenant-specific dashboards, documentation, presentation.
    \item \textit{Milestone:} Submit final capstone deliverables.
\end{itemize}

\section{Success Metrics \& Evaluation}

\subsection{Assessment Framework}
\begin{itemize}
    \item \textbf{Weekly Lab Completion (30\%):} Quality of microservice and polyglot persistence implementations, tests, and monitoring, evaluated via mentor reviews and Knowledge Checks.
    \item \textbf{Capstone Project (50\%):}
        \begin{itemize}
            \item \textit{Functionality (15\%):} Reliable polyglot document processing with microservices.
            \item \textit{Architecture Quality (15\%):} Scalability, maintainability, tenant isolation in microservices and polyglot databases.
            \item \textit{Code Quality (10\%):} Clean code, test coverage.
            \item \textit{Performance (10\%):} Meeting multitenant SLAs with optimization.
        \end{itemize}
    \item \textbf{Final Presentation (20\%):} Clarity in articulating microservice architecture, polyglot persistence, testing, and monitoring.
\end{itemize}

\subsection{Evaluation Criteria (Aligned with Program Pillars)}
\begin{itemize}
    \item \textbf{Technological Mastery:} Depth in EDD, Big Data, AI, microservices, polyglot persistence, testing, and monitoring.
    \item \textbf{Architectural Integration:} Design of cohesive multitenant microservice systems with Cassandra and PostgreSQL.
    \item \textbf{Production Readiness:} Focus on performance, security, scalability, testing, and monitoring.
    \item \textbf{Problem-Solving:} Innovation in polyglot and microservice challenges.
    \item \textbf{Communication:} Effective collaboration and presentation of results.
\end{itemize}

\section{Instructional Methodology}

\subsection{Learning Approach}
\begin{itemize}
    \item \textbf{Theory Sessions (43\%):} Workshops on concepts, microservices, polyglot persistence, testing, and monitoring.
    \item \textbf{Hands-On Labs (50\%):} Practical multitenant scenarios with microservices and polyglot storage.
    \item \textbf{Wrap-Ups (7\%):} Daily reflection and Q\&A.
    \item \textbf{Mentorship:} Weekly 1:1 sessions with architects.
    \item \textbf{Collaborative Learning:} Pair programming, group reviews.
\end{itemize}

\subsection{Delivery Model}
\begin{itemize}
    \item Expert-led workshops by Sami Mbarki and team
    \item Guided labs with microservice architecture, polyglot persistence, testing, and monitoring reviews
    \item Peer learning via collaborative problem-solving
    \item Continuous feedback and iterative improvement
    \item Production-like environment with enterprise tools
\end{itemize}

\subsection{Support for Diverse Learners}
\begin{itemize}
    \item \textbf{Pre-Course Refreshers:} Optional modules on Kafka, Spark, Cassandra, PostgreSQL, and microservices.
    \item \textbf{Advanced Tracks:} Deep dives into microservice optimization, polyglot persistence, and monitoring.
    \item \textbf{Knowledge Checks:} Daily quizzes on microservices, polyglot persistence, and testing.
    \item \textbf{Milestone Reviews:} Weekly capstone progress reflections.
\end{itemize}

\section{Terminology}
\begin{itemize}
    \item \textbf{Multitenancy:} Designing systems to securely isolate and process data for multiple clients (tenants) within a shared infrastructure.
    \item \textbf{Event-Driven Design (EDD):} Architecture using events to trigger and communicate between decoupled services.
    \item \textbf{Microservices:} Small, independently deployable services that handle specific business functions, communicating via APIs or events.
    \item \textbf{Polyglot Persistence:} Using multiple databases (e.g., Cassandra, PostgreSQL) to optimize for different data needs.
\end{itemize}

\end{document}